\documentclass[journal]{IEEEtran}
\usepackage{graphicx} % Required for inserting images

\title{G.A.R.D.E.N.I.R.S.}
\author{Nicholas Chitty - Electrical Engineer
\\Brendan College - Computer Engineer
\\Scott Pierce - Optical Engineer
\\Justin Pham-Trinh - Electrical Engineer}
\date{February 2023}

\begin{document}

\maketitle

\begin{center}
   \large Dept. of Electrical and Computer Engineering, University of Central Florida, Orlando, 
   Florida, 32816-2450
\end{center}

\begin{abstract}
   This paper will show the application of near infrared spectroscopy 
and how it can measure electromagnetic waves from the emission of soil. Near infrared spectroscopy 
is an absorption spectroscopy method that can help determine the chemical composition of a 
substance through the radiation the substance gives off. Soil itself is a mixture of organic and 
inorganic substances that all together directly contribute to a garden’s environment. We are 
starting with soil with unknown qualities, so comparisons will be made between our soil and soil 
of known qualities to match and ensure that our plant is in a healthy and suitable environment. 
\end{abstract}

\section{Introduction}
\IEEEPARstart{I}{n} the past year, we have seen a great increase in remote sensing, wireless 
communication, API integration, and so much more. All of which have been made more available and 
economical. The internet has also seen its share of "DIY" projects and its continuing growth.

In the agriculture industry, there have been new advancements in technology with high performance 
water distribution, network communication, and remote sensing. This research is intended to advance
the field by producing a system that can maintain a suitable environment for a plant to grow. In the 
environment, there will be sensors that will help modify the conditions within the environment. In 
addition to this system, it will feature a web interface for notifications and the ability for the 
user to set settings.

Monitoring soil isn't always the most fun or the easiest task, because things could get dirty or we 
might forget about our plant. This project will feature an on-the-rise technology in the form of a 
spectrophotometer. Smart systems nowadays are big learning machines that are constantly aware of its 
surroundings. For a smart agricultural system, it would need to determine variables such as moisture 
levels, nutrient content, pH levels, and so much more. This paper will introduce near infrared 
spectroscopy as another method to soil sensing that may prove to be more beneficial over traditional 
products or techniques.

This project will also present other fiedls such as system controls, power, and web, all to provide 
a "set it and forget it" home gardening experience. A home gardening experience where the garden bed 
can communicate with the user and the user can provide instruction of what to do. Like any smart 
device too, the internet plays an important role in making this as hands-off as it can be. There are 
many different communication protocols such as Bluetooth, Zigbee, Thread, and even short-range/long-
range protocol, but for this project we decided to go with WiFi because of its decreased bandwidth and 
that we aren't expecting to produce or receive large amounts of data.



\end{document}