\documentclass[journal]{IEEEtran}
\usepackage{amsmath}
\usepackage{gensymb}
\usepackage{graphicx} % Required for inserting images
\usepackage{tabularx}
\usepackage{listings} 
\usepackage{ltablex}
\usepackage{upgreek}
\usepackage{multirow}
\usepackage{hyperref}

\begin{document}

\begin{center}
   \Large Garden Automated Rain/Daylight Executed by Near-Infrared Sensing \break

   \large Nicholas Chitty, Brendan College, Scott Pierce, Justin Pham-Trinh \break

   \large University of Central Florida, Dept. of Electrical and Computer Engineering, Orlando,
   Florida, 32816-2450
\end{center}

\begin{abstract}
   This paper will show the application of near infrared spectroscopy 
and how it can measure electromagnetic waves from the emission of soil. Near infrared spectroscopy 
is an absorption spectroscopy method that can help determine the chemical composition of a 
substance through the radiation the substance gives off. Soil itself is a mixture of organic and 
inorganic substances that all together directly contribute to a garden’s environment. We are 
starting with soil with unknown qualities, so comparisons will be made between our soil and soil 
of known qualities to match and ensure that our plant is in a healthy and suitable environment. 
\end{abstract}

\section{Introduction}
\IEEEPARstart{I}{n} the past year, we have seen a great increase in remote sensing, wireless 
communication, API integration, and so much more. All of which have been made more available and 
economical. The internet has also seen its share of ``DIY" projects and its continuing growth.

In the agriculture industry, there have been new advancements in technology with high performance 
water distribution, network communication, and remote sensing. This research is intended to advance
the field by producing a system that can maintain a suitable environment for a plant to grow. In the 
environment, there will be sensors that will help modify the conditions within the environment. In 
addition to this system, it will feature a web interface for notifications and the ability for the 
user to set settings.

Monitoring soil isn't always the most fun or the easiest task, because things could get dirty or we 
might forget about our plant. This project will feature an on-the-rise technology in the form of a 
spectrophotometer. Smart systems nowadays are big learning machines that are constantly aware of its 
surroundings. For a smart agricultural system, it would need to determine variables such as moisture 
levels, nutrient content, pH levels, and so much more. This paper will introduce near infrared 
spectroscopy as another method to soil sensing that may prove to be more beneficial over traditional 
products or techniques.

This project will also present other fields such as system controls, power, and web, all to provide 
a "set it and forget it" home gardening experience. A home gardening experience where the garden bed 
can communicate with the user and the user can provide instruction of what to do, but also at the 
core, this is a microcontroller project that anybody can do with just slight knowledge and understanding
of digital communications. Like any smart device too, the internet plays an important role in making 
this as hands-off as it can be. There are many different communication protocols such as Bluetooth, 
Zigbee, Thread, and even short-range/long-range protocol. For this project we decided to go with WiFi 
because of its decreased bandwidth and that we aren't expecting to produce or receive large amounts 
of data. 

An important aspect to a lot of devices as well is data storage and web usage. This is important so 
that we can look back on previous information and make observations and conclusions. With our 
microcontroller, the web system is going to communicate with it through transmission control protocol.
Transmission control protocol is a standard on how to establish and maintain a network connection to 
exchange data. The web system will also have 16gb database to store data and have the ability to support 
multiple garden beds in a scaled solution. As mentioned before, the web system will have a feature to 
allow the user to set settings but also read the data that is stored. Lastly, an important factor to 
any garden is the weather and knowing when it might be cold, hot, or even rain outside. As a part of 
our web system, it will communicate using HTTP requests with a weather service to receive updates on 
the weather.

Solar power has been a growing source of energy for the past years and still continues to be with new 
developments and breakthroughs with solar technology. A lot of systems nowadays are solar powered but 
these products are not constantly in use, they have to turn of eventually. When the product is off the 
solar panel can still collect energy which is stored in a battery for conservation. In this paper, we 
will demonstrate our system running independently, battery powered but charged through solar panels.

In all, this paper will present how near infrared spectroscopy can be used to monitor a plant's environment 
and notify users with information of its health. Generating and guiding electromagnetic waves into a 
housing, scanning the substance, and then converting this optical power into a voltage that can be 
read and analyzed. It is then the microcontroller will communicate with the web system, storing the 
information and deciding on if anything needs to be done and notifying the user. 

\section{Controller Subsystem}

% Why the MCU connects to the internet
In order for our system to be as self and power efficient as possible from an end-user perspective, our team decided to use a low-power, Internet-of-Things (IoT)-focused wireless microcontroller. To make the process of operating our product as hands-off as possible to end-users, the microcontroller will operate in Access Point (AP) mode to serve information to the user's mobile device. We do not expect our product to produce or receive large amounts of data, so the decreased bandwidth of a WiFi-enabled product being beyond the outdoor walls of a building is not a significant drawback to our application.

% How the MCU connects to the internet (local network, LAN -> NAT -> WAN, TCP stack)
\subsection{CC3220 Overview}
The Texas Instruments CC3220-series (hence referred to as the "CC3220", the "MCU", or the "microcontroller") of microcontrollers are WiFi-enabled chips with an ARM Cortex-M4 central processor and a WiFi network processor, along with many useful peripherals and power management modules. This series of processors is delivered alongside a software development kit (SDK) provided by Texas Instruments to ease the development of IoT applications.

The CC3220's WiFi network processor (NWP) supports 802.11b/g/n, SmartConfig provisioning, IPv4 and IPv6. The NWP also as the ability to host an internal HTTP/HTTPS server, and contains its own filesystem.

\subsection{Connection}
Our product shall be able to broadcast a WLAN in AP mode. This will allow the user to connect to the product's SSID. The user would then be directed to a web portal hosted by the MCU's internal HTTP server, where they will be able to view information and telemetry related to the product. Hosting a web interface would allow unanimous adaptation of our product for home users. The system shall be able to broadcast a WLAN in AP mode with a nominal signal strength of 6 dBm measured at the antenna. The network shall adhere to the standards of 802.11b, g, or n. 

% Parameters for connection and how often it tries
\subsection{Weather API Implementation}
As a stretch goal, the user will be able to configure the MCU in Station mode to allow it to connect to the user's home WLAN's SSID as a client. The user would still be able to access the MCU's web portal and see the same information as they had before. In addition, the MCU would be able to access an API to receive weather information and inform the user of recommended actions pertaining to their garden bed (e.g. if the system determines there will be freezing temperatures overnight, it may suggest to the user to cover the garden bed with a sheet or towel).

% Any over-the-air updates?
\subsection{Over-the-Air Updates}
At this time, our team does not intend to provide a method for over-the-air
updates (OTA), however, this is a stretch goal that may be completed in the
future.

\subsection{Startup and Shutdown} Startup and shutdown will occur in a timely manner (within seconds). Abrupt shutdown shall not damage any components of the system, and the system will be able to restart to its previous state without any input from the user. If the system freezes or crashes, its internal watchdog timer will automatically reboot the device.

\subsection{Data} The data available to the user shall be the OH-levels and common plant nutrient levels as determined by the spectral analysis of the product. The user shall also be able to see and configure the network settings of the product.

\subsection{Low Power Modes} The system shall run in active mode between sunrise and sunset, while the battery has 40-100\% calculated charge remaining. The system shall run in low power mode between sunrise and sunset while the battery has 10-40\% calculated charge remaining, and between sunset and sunrise while the battery has 10-100\% calculated charge remaining. The system shall run in critical power mode while the battery has 5-10\% calculated charge remaining. The system shall be in shutdown mode while the battery has 0-5\% calculated charge remaining. The system shall be able to switch power modes within 5 seconds of the interrupt being triggered.

% Libraries
\subsection{Libraries}
It should be expected that the C++ standard library (as defined in C++14) will be used, as well as the POSIX library, along with the Texas Instruments SimpleLink CC32xx SDK.

\subsection{Sensing} The system shall monitor and perform analog-to-digital conversion of the spectral values sensor values at least every 15 minutes while in active mode. The system will use I2C to talk to the analog-to-digital converter (ADC) located in the sensing subsystem. 

% How the MCU gets data from sensors (ADC)
\subsection{Analog-to-digital Conversion}
The sensing subsystem contains a Texas Instruments ADS 7142 ADC. The ADS 7142 is an IoT-focused nanowatt ADC with 2 external channels, an I2C interface, a sampling frequency of 140ksps, and an effective resolution of 16 bits in High-precision Mode. This ADC will be able to measure between 0 and 3.3 V from the output of the sensing subsystem's  photodiode op-amp circuit.

% How the MCU gets data from sensors (ADC, cont.)
It is expected that each photodiode's circuit will provide a voltage of between 0 and 3.3 V. The 16-bit ADC provides for an input of the same range. Therefore, the resolution of the ADC is calculated below:
\begin{equation}
    \frac{(3.3 - 0)\,\mathrm{V}}{2^{16}\,\mathrm{steps}} =
    50.35\,\mathrm{\mu V}/\mathrm{step}
\end{equation}
These steps will be used to measure OH composition and nutrients in the soil. The MCU will directly control GPIO and bit-bang values to the stepper motor controlling the position of the photodiodes allowing them to measure the full range of spectral values.

% What kind of development model?
\subsection{Development Model}
An Agile development model was used to program the control subsystem and its various modules. Code reviews were performed on an as-needed basis by a convening of members of the MCU subsystem and the web subsystem teams. Texas Instruments Code Composer Studio v12 was used to program, compile (via TI ARM compiler v20), and debug the C++-based project. GitHub was used as a repository for the project, using GNU Git for version control.

% Power
\subsection{Power}
The MCU will receive regulated 3.3, 5, and 12 V power from from the power subsystem. This power will be routed to various modules and other subsystems.

% LEDs
\subsection{LEDs} The system shall be able to make use of its onboard LEDs for notifying the user or developers of system states. When the system is in active mode, the green LED shall be solidly illuminated. When the system is in low power mode, the green LED shall flash 1 time for a period of 0.5 seconds, every 2 seconds. When the system is in critical power mode, the red LED shall flash 2 times for a period of 0.5 seconds per flash, seperated by 1 second between each flash, every 30 seconds. When the system is in shutdown mode, no LEDs shall be active. When the system is starting up, the green and red LED shall be solidly illuminated until the startup sequence is completed and the system transitions into a different power mode.

% PCB
\subsection{Printed Circuit Board} Our controller subsystem was developed and implemented on a Texas Instruments LaunchPad development board. An initial control printed circuit board (PCB) was designed, fabricated, and assembled. The chip version of the CC3220 (CC3220S) was used requiring a 4-layer impedance-controller PCB, with waveguide and external antenna, as well as oscillators, numerous bypass capacitors, and inductors. A revised PCB was designed using the module version of the CC3220 (CC3220MODAS). The PCB design was simplified, requiring only a few bypass capacitors instead of the numerous other components to support the microcontroller found on the first revision of the board.

\subsection{Goals} \label{sec:goals}
The following are the goals the team set for themselves with this project. Starting with non-
negotiable goals:
\begin{enumerate}
   \item A spectrometer that operates in the 400nm to 1700nm band with a spectral resolution less
         than 50nm and signal-to-noise ratio greater than 2.
   \item The ability to automatically feed water into the garden bed.
   \item Serve data to the user.
\end{enumerate}
Moving on, the team had goals they hoped to accomplish:
\begin{enumerate}
   \item Power the garden bed completely from solar and battery power.
   \item Serve data to the user via a website.
\end{enumerate}
\section{High-Level Design}
Based on the goals listed in \autoref{sec:goals} this section will discuss the high-level thought
process the team undertook in achieving those goals.
\subsection{Spectrometer} 
In most instances, spectrometers are designed using charged couple devices in an array which some
component such as a prism or diffraction grating will shine upon breaking the source into its
components. This approach would be prohibitively expensive for such a consumer-oriented garden
bed. The solution was to use either a photodiode or photoresistor to measure the magnitude of the
of optical power in a given area. This sensor could be moved in order to give the effect of shining
the source on an array with the only major downside being the duration needed to scan. 
\section{Major Components} \label{sec:major-components}
The components found in this section make up the logical boundaries between systems and their
subsystems. 
\section{Hardare Detail}
In \autoref{sec:major-components}, a high level overview of components and features were given. This
section will discuss the implementation details of the chosen components.
\section{Software Detail}
The microntroller serves as the glue holding this design together. In this section the means of
integrating the various subsystems via software will be discussed.
\end{document}